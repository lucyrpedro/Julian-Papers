\documentclass[a4paper,14pt]{extarticle}
\usepackage[hmargin=2cm,vmargin=2cm,bmargin=3cm]{geometry}
% \usepackage{geometry}
\usepackage{amsfonts}
\usepackage{amssymb}
\usepackage{marvosym}
\usepackage{color}
\usepackage{comment}

\newcommand{\espaco}{\hspace{16pt}}

\title{\espaco\\[3cm] \huge Julian Kunkel's Papers\\[5cm]}

\author{\LARGE Title and Abstract\\[3cm]}

\date{\today}
		
\begin{document}

\maketitle

\parskip = 25pt
\baselineskip = 20pt

\clearpage

\section{HDTrace - A Tracing and Simulation Environment of Application and System Interaction}

\espaco
HDTrace is an environment which allows to
trace and simulate the behavior of MPI programs on a
cluster. It explicitly includes support to trace internals
of MPICH2 and the parallel file system PVFS. With
this support it enables to localize inefficiencies, to conduct research on new algorithms and to evaluate future
systems. Simulation provides upper bounds of expected
performance and helps to assess observed performance
as potential performance gains of optimizations can be
approximated.

In this paper the environment is introduced and
several examples depict how it assists to reveal internal behavior and spot bottlenecks. In an example with
PVFS the inefficient write-out of a matrix diagonal
could be either identified by inspecting the PVFS server
behavior or by simulation. Additionally the simulation
showed that in theory the operation should finish 20
times faster on our cluster - by applying correct MPI
hints this potential could be exploited.\\[1cm]

{\bf Comments}

\clearpage

\section{Towards an Energy-Aware Scientific I/O Interface}

\espaco
Intelligently switching energy saving modes
of CPUs, NICs and disks is mandatory to reduce the
energy consumption. 

Hardware and operating system have a limited perspective of future performance demands, thus automatic
control is suboptimal. However, it is tedious for a developer to control the hardware by himself.

In this paper we propose an extension of an existing I/O interface which on the one hand is easy to use
and on the other hand could steer energy saving modes
more efficiently. Furthermore, the proposed modifications are beneficial for performance analysis and provide even more information to the I/O library to improve performance.

When a user annotates the program with the proposed interface, I/O, communication and computation
phases are labeled by the developer. Run-time behavior is then characterized for each phase, this knowledge
could be then exploited by the new library.\\[1cm]

{\bf \large Comments} 

\begin{itemize}

\item Several best practices are realized within the ADIOS library to increase usability and performance, for instance aggressive write-behind is performed, and MPI collectives transfer file information to decrease the burden on metadata servers.

\item Available modules include NetCDF, HDF5, MPI (collective or independent), POSIX and several asynchronous staging modules.

\end{itemize}

\clearpage

\section{Simulating Parallel Programs on Application and System Level}

\espaco
Understanding the measured performance
of parallel applications in real systems is difficult - with
the aim to utilize the resources available, optimizations
deployed in hardware and software layers build up to
complex systems. However, in order to identify bottlenecks the performance must be assessed.

This paper introduces PIOsimHD, an event-driven
simulator for MPI-IO applications and the underlying
(heterogeneous) cluster computers. With the help of
the simulator runs of MPI-IO applications can be conducted in-silico; this includes detailed simulation of collective communication patterns as well as simulation of
parallel I/O. The simulation estimates upper bounds
for expected performance and helps assessing observed
performance.

Together with HDTrace, an environment which allows tracing the behavior of MPI programs and internals of MPI and PVFS, PIOsimHD enables us to localize inefficiencies, to conduct research on optimizations
for communication algorithms, and to evaluate arbitrary and future systems. In this paper the simulator is
introduced and an excerpt of the conducted validation
is presented, which demonstrates the accuracy of the
models for our cluster.\\[1cm]

{\bf \large Comments} 

\clearpage


\end{document}
